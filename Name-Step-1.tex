\documentclass[12pt]{article}
\usepackage[utf8]{inputenc}
\usepackage[T1]{fontenc}
\usepackage{amsmath,amsfonts,amssymb}
\usepackage{graphicx}
\usepackage{a4wide}
\title{Reconstructed abstract of the paper ``Solving olympiad geometry without human demonstrations''}
%\author{not specified, not necessary here}
\date{}
\begin{document}
\maketitle

\begin{abstract}
%The abstract is limited to 600 characters. It may contain a wide-range field of the investigated problem, narrow problem to focus on, features and conditions of the problem, the idea of the suggested solution, the novelty and alternative solutions to compare with, application to illustrate with.
AlphaGeometry, a novel AI system, advances mathematical reasoning by solving complex geometry problems at a level comparable to human Olympiad gold medalists. Tested on 30 Olympiad problems, AlphaGeometry solved 25, outperforming previous AI systems that solved only 10. This breakthrough leverages a neuro-symbolic approach, combining neural language models and symbolic deduction engines, enhanced by the generation of 100 million synthetic training examples. Verified and praised for its verifiable, human-readable solutions, AlphaGeometry marks a significant stride in AI-driven mathematical discovery.
\end{abstract}
\paragraph{Keywords:} AlphaGeometry, Artificial Intelligence, Mathematical Reasoning, Geometry Problems, Neuro-Symbolic Systems, Synthetic Data Generation, Olympiad-Level Mathematics, Symbolic Deduction, Neural Language Models, AI in Mathematics.

\paragraph{Highlights:}
\begin{enumerate}
\item AlphaGeometry achieves human-like proficiency in solving complex Olympiad-level geometry problems.
\item The system successfully solved 25 out of 30 problems, significantly outperforming previous AI methods.
\item Employs a neuro-symbolic approach, integrating neural language models with symbolic deduction engines.
\item Trained on 100 million synthetic examples, enhancing its problem-solving capabilities.
\item Provides solutions that are machine-verifiable yet human-readable, bridging the gap between automated and manual problem solving.
\item Marks a significant advancement in AI-assisted mathematical discovery and proof validation.
\end{enumerate}

\paragraph{Motivation:}
~\\
% Explain the reasons you chose the paper~\cite{9095246}. The reasons shall not be personal. They shall enlighten the novelty, impact, or contributions.
The paper on AlphaGeometry was selected due to its breakthrough in AI-driven mathematical reasoning, an area ripe with potential yet fraught with challenges. By successfully tackling complex geometry problems at an Olympiad level, the research demonstrates a novel and impactful neuro-symbolic approach that synergizes neural language models with symbolic deduction. This integration not only pushes the boundaries of what AI can achieve in specialized domains but also addresses critical aspects of solution interpretability and verifiability. With the potential to transform educational methodologies and inspire future technological innovations, AlphaGeometry embodies a significant leap forward in harnessing AI for mathematical discovery and understanding.

\bibliographystyle{unsrt}
\bibliography{Name-theArt}
\end{document}